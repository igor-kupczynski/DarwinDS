\label{user-manual}

\section*{A graphical user interface for DARWIN}

\subsection*{Starting the GUI application}

\begin{figure}[htb]
  \centering
  \includegraphics[scale=0.7]{img/manual/01_main_screen}
  \caption{The main screen of DARWIN}
  \label{manual_01_main}
\end{figure}

\subsection*{Solving exemplary problems}

DARWIN offers an intuitive and easy to use interface for solving MMO
problems. The main window is presented in figure~\ref{manual_01_main}. To
solve a~problem one need to execute following steps:

\begin{enumerate}
\item Click the ``Open'' button in the main window.
\item Choose a problem file using provided file selector
  (fig.~\ref{manual_02_problem_selector}). The problem has to be described in
  the DARWIN .mod (model) file.
\item The main window now shows a name of the problem
  (fig.~\ref{manual_03_selected}). Click ``Solve''.
\item A window presenting generated solutions is presented
  (fig.~\ref{manual_04_mark_as_good}). Mark preferred solutions as ``good''
  using checkboxes in the ``is good'' column.
\item A window with solution details can be invoked by selecting a~solution
  and clicking the ``Solution details'' button (fig.~\ref{manual_05_dec_var}).
\item After selecting the solutions an evolutionary optimization begins
  (fig.~\ref{manual_06_evo_loop}). New solutions are generated and one can
  mark ``good'' solutions from a new set again.
\item When generated solutions are satisfactory then one can stop the
  algorithm. Just select no solutions and click ``Mark as
  good''. Fig.~\ref{manual_07_finish}.
\end{enumerate}

\begin{figure}
  \centering
  \includegraphics[scale=0.7]{img/manual/02_problem_selector}
  \caption{Selecting a problem to solve}
  \label{manual_02_problem_selector}
\end{figure}

\begin{figure}
  \centering
  \includegraphics[scale=0.7]{img/manual/03_problem_selected}
  \caption{Starting the algorithm}
  \label{manual_03_selected}
\end{figure}

\begin{figure}
  \centering
  \includegraphics[scale=0.7]{img/manual/04_marking_solutions}
  \caption{Marking the ``good'' subset of generated solutions}
  \label{manual_04_mark_as_good}
\end{figure}

\begin{figure}
  \centering
  \includegraphics[scale=0.7]{img/manual/05_solution_details}
  \caption{Decision variables for a given solution}
  \label{manual_05_dec_var}
\end{figure}

\begin{figure}
  \centering
  \includegraphics[scale=0.7]{img/manual/06_evolutionary_loop}
  \caption{An evolutionary optimization is performed}
  \label{manual_06_evo_loop}
\end{figure}

\begin{figure}
  \centering
  \includegraphics[scale=0.7]{img/manual/07_end_of_run}
  \caption{Finish the problem analysis}
  \label{manual_07_finish}
\end{figure}

\clearpage{}
\subsection*{Advanced options}

The DARWIN GUI offers additional options to customize the process. One can
modify the algorithm parameters in configuration dialog
(fig.~\ref{manual_08_options}). To open it click the ``Options'' dialog in the
main window. The options are described below.

\begin{table}[htb]
  \centering
  \begin{tabular}{l p{3.5cm} p{6.5cm} l l}
    \hline
    Tab & Option name & Description & Default value \\
    \hline
    \hline
    \multirow{5}{*}{Main}
    & The number of solutions & The number of solutions in a population. & 30 \\
    & The number of scenarios & The number of scenarios on which the solutions will be evaluated.  & 30 \\
    & The number of generations & The number of generations in the interior loop. & 30 \\
    & Percentiles & Which percentiles are meaningful to the decision maker. & 1.0, 25.0, 50.0  \\
    & Use an average in quantiles & Whether an average in quantiles should be used instead of
    the maximum value). & false  \\
    \hline
    \multirow{3}{*}{The Algorithm}
    & Use All Rules instead of DomLem & Should the All Rules algorithm be used
    instead of the default one & false \\
    & The DomLEM confidence level & A level of confidence that the rules generated by the
    DomLem algorithm should at least have. & 0.6 \\
    & Compare using the supposed utility function & If the evolutionary algorithm should use the
    supposed utility instead of a rule-based score. & false \\
    \hline
    \multirow{4}{*}{Fine Tuning} 
    & Delta & The decay of rule weight (see \ref{idea-algo}). & 0.1  \\
    & Gamma & The coefficient of elitism. The higher the gamma, the
    higher the probability of choosing a solution with a~higher rank as
    a~parent. & 2.0  \\ 
    & Eta & The initial mutation probability.  & 0.5    \\
    & Omega & The decay rate of the mutation probability. & 0.1  \\
    \hline
    \multirow{5}{*}{Reports} 
    & The reports directory & A directory where reports should be saved &
    ./reports  \\
    & Save the evolutionary report & If the evolutionary report should be
    saved & false \\
    & Save the decision maker's report & If the DM's report should be
    saved & false \\
    & The rules directory & A directory where rules should be saved &
    ./rules  \\
    & Save rules & If the decision rules generated during a run should be
    saved & false \\
    \hline
    \multirow{1}{*}{GUI Parameters}
    & Digits after a dot & A number of digits that should be displayed after a
    dot & 2 \\    
    \hline
  \end{tabular}
\end{table}


The window presenting list of generated solutions offers additional
features. A list of solutions can be sorted on a given criterion by clicking
the column header (fig.~\ref{manual_09_sorton}). Another useful option is a
history of solutions. The user can review solutions generated in earlier
iteration, return to them change his or her selections. This is possible using
the history toolbar (see\ref{manual_10_history}).

\begin{figure}
  \centering
  \includegraphics[scale=0.7]{img/manual/08_options}
  \caption{DARWIN configuration options}
  \label{manual_08_options}
\end{figure}

\begin{figure}
  \centering
  \includegraphics[scale=0.7]{img/manual/09_sorton}
  \caption{Sorting solutions on a given criterion}
  \label{manual_09_sorton}
\end{figure}

\begin{figure}
  \centering
  \includegraphics[scale=0.7]{img/manual/10_history}
  \caption{Reviewing the history of generated solutions}
  \label{manual_10_history}
\end{figure}


\section*{The DARWIN .mod file format}

\section*{Compiling DARWIN from the source code}

\section*{The Experiment framework} 

%%% Local Variables: 
%%% mode: latex
%%% TeX-master: "main"
%%% End: 
