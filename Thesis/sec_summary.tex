In the paper a~novel approach to multi-objective optimisation was
presented. The DARWIN method is an interactive procedure utilising the
evolutionary algorithm to optimize a~population of solutions. The idea of
DARWIN was first proposed by Salvatore Greco, Benedetto Matarazzo and Roman
Słowiński in~[ref]. However in this paper first implementation and numerical
results are provided.

The novelty of the method comes from utilizing an IMO process along with the
EMO procedure. DARWIN not only optimises the population of solutions but also
drives the optimization towards regions preferred by the decision maker. In
order to do it, the preference information has to be gathered. This is done by
asking the DM series of simple questions. a~list of feasible solutions is
given and he or she is asked to indicate the ``good'' ones among them. The
procedure uses dominance-based rough set approach and the DomLem algorithm to
obtain a~set of \textit{``if \dots, then \dots''} decision rules. The DARWIN
is a~first MOO technique using the decision rules.

The condition part of each rule corresponds to a~dominance cone in the
objective space build on a~subset of objectives. If a~given solution matches
the conditional part of the rule it is considered ``good'' with respect to
this rule. The higher the number of rules matched, the higher the fitness
score in the evolutionary optimisation, thus the bigger the chance to
``survive'' and advance to a~next generation.

The DARWIN allows one to use intervals of possible values in the problem
formulation. Multiple scenarios are then tested. Objective space is
transformed --- it is no longer possible to provide a~value of an
objective. One has to reason in terms of meaningful quantiles of each of the
objectives. This allows to take into account decision maker's attitude towards
risk.

Two characteristics of the DARWIN ensures robustness of the generated
solutions. Firstly, each solution is tested on multiple scenarios of
uncertainty, so its characteristics are known even considering fluctuations in
the problem parameters. Secondly, the decision rules generated by the DRSA
framework are immune to inconsistencies in decision maker's
choices. Therefore, the algorithm can withstand a~noise in his or her decision
and still guide the search towards preferred regions.

Performed computational experiments confirm the author intuition that the
method can be used to solve a~class of multi-objective optimization
problems. If there is no uncertainty involved --- the exact values of all
problem coefficients are known --- the resulting solutions is not further than
$10\%$ from the optimal one. If the uncertainty is allowed comparison to the
optimal solution is impossible, because it can not be provided to the problem
that is not a~well-defined. However, comparison of the evolutionary
optimization based on the DRSA decision rules and the one where supposed
utility function is used can be made. The behavior of both optimizations is
similar and the conclusion is that the preference information extracted from
the DM guides the algorithm to the same regions where the supposed utility
optimization. The results also showed that the DARWIN is immune to changes in
parameters and, moreover, to the noise in the decision maker's choices.

The implementation was done in the Scala programming language. The language
runs on top of the Java Virtual Machine (JVM). The JVM is a well-tested
platform with enormous popularity in an enterprise class of solutions. The
platform is available on all of the popular hardware and software
architectures. This ensures portability of the code as well of a stability of
the developed solution. Creating a~component for a java platform increases the
possibility of reuse the component in further projects and solutions.

Future work may include performing more computational experiments. One could
prepare a~catalogue of problems along with a~description of the DARWIN's
behavior and results that can be achieved on them. One could also modify the
evaluational optimization framework and check if it leads to a~different
behavior on some of the problems. For generating a set of decision rules the
DomLem algorithm is used, however one can check if switching the algorithm to
another one will lead to a~substantial changes in the results. Finally, one
could prepare an easy-to-use computer software for the analyst and the
decision-maker and, if possible, distribute it under open-source licence.

The goal of the thesis was to describe, implement and test the DARWIN
method. Implementation was prepared and experiments conducted. A
recommendations on parameter values, problems that can be solved and the
expected performance are given. To summarize, it can be stated that the goals
of the thesis have been achieved.

