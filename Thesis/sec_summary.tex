In the paper a~novel approach to multi-objective optimization was
presented. The DARWIN method is an interactive procedure utilizing the
evolutionary algorithm to optimize a~population of solutions. The idea of
DARWIN was first proposed by Salvatore Greco, Benedetto Matarazzo and Roman
Słowiński in~[ref]. However, in this paper the first implementation and
numerical results are provided.

The novelty of the method consists in utilizing an IMO process along
with the EMO procedure. DARWIN not only optimizes the population of
solutions, but also drives the optimization towards regions preferred
by the decision maker. In order to do it, the preference information
has to be gathered. This is done by asking the DM a series of simple
questions. A~list of feasible solutions is given and he or she is
asked to indicate the ``good'' ones among them. The procedure uses
dominance-based rough set approach and the DomLem or AllRules
algorithms to obtain a~set of \textit{``if \dots, then \dots''}
decision rules. DARWIN is a~first MOO technique using the decision
rules.

The condition part of each rule corresponds to a~dominance cone in the
objective space built on a~subset of objectives. If a~given solution matches
the conditional part of the rule, it is considered ``good'' with respect to
this rule. The higher the number of rules matched, the higher the fitness
score in the evolutionary optimization, thus the bigger the chance to
``survive'' and advance to the~next generation.

DARWIN allows one to use intervals of possible values in the problem
formulation. Multiple scenarios are then tested. Objective space is
transformed --- it is no longer possible to provide a~value of an
objective. One has to reason in terms of meaningful quantiles of each of the
objectives. This allows to take into account the decision maker's attitude
towards risk.

Two characteristics of the DARWIN method ensure robustness of the generated
solutions. Firstly, each solution is tested on multiple scenarios of
uncertainty, so its characteristics are known even considering fluctuations in
the problem parameters. Secondly, the decision rules generated by the DRSA
framework are immune to inconsistencies in the decision maker's
choices. Therefore, the algorithm can withstand a~noise in his or her decision
and still guide the search towards the preferred regions.

Performed computational experiments confirm the author's intuition that the
method can be used to solve a~class of multi-objective optimization
problems. If there is no uncertainty involved --- the exact values of all
problem coefficients are known --- the resulting solutions are not further
than $10\%$ from the optimal one. If the uncertainty is allowed a~comparison
to the optimal solution is impossible, because it can not be provided to the
problem that is not well-defined. However, a~comparison of the evolutionary
optimization based on the DRSA decision rules with the one where the supposed
utility function is used can be made. The behavior of both optimizations is
similar and the conclusion is that the preference information extracted from
the DM guides the algorithm to the same regions where the supposed utility
optimization. The results also showed that DARWIN is immune to changes in
parameters and, moreover, to the noise in the decision maker's choices.

The implementation was done in the Scala programming language. The language
runs on top of the Java Virtual Machine (JVM). The JVM is a well-tested
platform with enormous popularity in an enterprise class of solutions. The
platform is available on all of the popular hardware and software
architectures. This ensures portability of the code as well stability of the
developed system. Creating a~component for a java platform increases the
possibility of reusing the component in further projects and solutions.

\textcolor{red}{Future work may include performing more computational
  experiments. A~catalogue of problems along with a~description of DARWIN's
  behavior and results that can be achieved on them could be prepared. One
  could also modify the evaluational optimization framework and check if, for
  example, introducing other crossover operators leads to a~different behavior
  on some of the problems.} 

\textcolor{red}{ In the current implementation the Monte Carlo procedure are
  using a~uniform distribution of values in intervals.  An extension adding
  the possibility of selecting different distributions could be developed. }

The goal of the thesis was to describe, implement and test the DARWIN
method. Implementation was prepared and experiments conducted. Recommendations
on parameter values, problems that can be solved and the expected performance
are given. To summarize, it can be stated that the goals of the thesis have
been achieved.


%%% Local Variables: 
%%% mode: latex
%%% TeX-master: "main"
%%% End: 
