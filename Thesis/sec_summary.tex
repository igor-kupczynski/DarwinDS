In the paper a novel approach to multi-objective optimisation was
presented. The DARWIN method is an interactive procedure utilising the
evolutionary algorithm to optimize population of solutions. The idea of DARWIN
was first proposed by Salvatore Greco, Benedetto Matarazzo and Roman Słowiński
in~[ref]. However in this paper first implementation and numerical results are
provided.

The novelty of the method comes from utilizing an IMO process along with the
EMO procedure. DARWIN not only optimises the population of solutions but also
drives the optimization towards regions preferred by the decision maker. In
order to do it, the preference information has to be gathered. This is done by
asking the DM series of simple questions. A list of feasible solutions is
given and he or she is asked to indicate the ``good'' ones among them. The
procedure uses dominance-based rough set approach and the DomLem algorithm to
obtain a set of \textit{``it \dots, then \dots''} decision rules. The DARWIN
is a first MOO technique using the decision rules.

The condition part of each rule corresponds to a dominance cone in the
objective space build on a subset of objectives. If a given solution matches
the conditional part of the rule it is considered ``good'' with respect to
this rule. The higher the number of rules matched, the higher the fitness
score in the evolutionary optimisation, thus the bigger the chance to
``survive'' and advance to next generation.



