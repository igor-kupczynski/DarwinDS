Optimization is a~process of finding the best solution from a~set of available
alternatives. In the simplest case it involves only a~single objective. An
objective is a~problem's goal, whose value has to be maximized (or
minimized). An optimization problem consists of a~set of decision variables;
values of those variables affect the problem's objectives. A~domain of the
problem --- a~set of possible values that can be taken by the decision
variables --- may be subject to additional constraints. An assignment of
values to the decision variables is called a~solution. A~solution is feasible
if the values meet the constraints imposed on the problem. Therefore, the task
of optimization is to find a~feasible solution, which maximizes (minimizes)
the value of the problem's objectives.

Optimization is a~field of applied mathematics, which has gained importance
during and after the Second World War. The driver for developments in the
field are real-life, mainly military or industrial problems. They are often
large-scale and of high importance; solving a~problem of this kind usually
yields a~profit outranking the costs that one has to bear to employ a~formal
approach. To use optimization methods one has to formulate a~problem in a
mathematical way --- build a~model of the problem. Therefore, it is common to
make a~distinction between a~stakeholder --- a~decision maker (DM) having
expert knowledge in a~problem domain and an analyst --- a~mathematician or
a~scientist that will create the model and choose an appropriate optimization
technique.

If there is only one, single objective in the problem definition, then the
optimization usually results in a~one optimal solution which has the best
possible value of the goal. However, this is not the case when multiple goals
are considered. Usually the optimal value for one goal is far from being an
optimal for the other; for example, consider a~simple problem of choosing a
laptop computer to buy --- the one with the highest performance is not the
cheapest and the cheapest will probably not offer a~state-of-the-art
performance. No trade-offs between objectives can be assumed \textit{a
  priori}. This is because the importance of each goal may be different for
different decision makers --- someone can choose a~cheap computer, while the
other will opt for the highest performance available. However, one can
indicate a~non-dominated set of solutions --- the Pareto-frontier of the
problem. Informally, a~solution is Pareto-efficient if a~value of any of its
criteria can not be improved without worsening the other values. In the laptop
example it could be a~list of the cheapest computers for different
performances.

Multi-Objective Optimization (MOO) methods are usually sophisticated, with
many parameters for the analyst to fine-tune them. However, the decision maker
is usually not interested in the details of the method, but only in a~single
recommendation, possibly along with a~justification. The analyst will set up
the method and its parameters based on his or her intuition and on a~research
carried out earlier. An experienced analyst can usually decide what
parameters' values should be used, but still it may be impossible to set the
values precisely to the best possible options. If one can change parameters a
bit and the resulting solution is similar to the one acquired before, then the
solution is called robust. The same applies if the problem model can not be
formulated precisely --- for example, it contains a~value that may be only
estimated, like the future price of a~raw material. The solution should be
resistant to small fluctuations in the problem's model parameters. The
robustness in MOO context is the ability to withstand changes in the
parameters and in the problem formulation; it is a~very important quality of
any MOO technique.

To give final recommendation instead of the Pareto-frontier one has to engage
the decision maker in the process. The method has to be interactive in order
to gather the DM's preferences. It can be done by showing exemplary feasible
solutions and asking the decision maker to rank them or simply by asking about
the inter-criteria trade-offs. These preferences are used to guide the search
of the solution space in the directions desired by the DM. An optimization
technique involving interaction with the decision maker is called the
interactive multi-objective optimization (IMO) technique.

The algorithm is a~well-defined list of instructions for completing a
task. Several researchers suggested that principles of the evolution ---
particularly, the concept of population and survival of the fittest
individuals can be a~good model of operation for multi-criteria optimization
algorithms. Methods using these principles are called the evolutionary
algorithms (EAs) while the whole field of research is the evolutionary
multi-objective optimization (EMO).

In this paper the author presents the DARWIN method. DARWIN is an acronym for
Dominance-based rough set Approach to handling Robust Winning solutions in
INteractive multi-objective optimization. DARWIN is an algorithm proposed by
Salvatore Greco, Benedetto Matarazzo and Roman Słowiński; dedicated for
solving multi-objective optimization problems. It interacts with the decision
maker in order to infer his or her preferences. The preferences are then
stored in the form of decision rules guiding the optimization process. An
evolutionary algorithm is used as an engine for optimization. Therefore DARWIN
combines IMO and EMO paradigms. It allows an analyst to model uncertainty in
the problem definition, thus generating robust solutions.

\subsection{Goal and scope of the thesis}

The thesis consists of six chapters. Firstly, the theoretical background is
presented. A~more detailed description of the DARWIN algorithm follows in the
chapter~\ref{darwin-the-idea}. The chapter~\ref{darwin-implementation}
discusses an implementation on an IBM-PC class computer. Experiment results
are shown and discussed in the chapter~\ref{exp-results}. Finally, areas of
further research are indicated along with conclusions and recommendations
about the method.

%%% Local Variables: 
%%% mode: latex
%%% TeX-master: "main"
%%% End: 
