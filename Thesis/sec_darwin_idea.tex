The basic idea of Darwin method was introduced in~ \cite{GMS09}. This idea will
be described in the following paragraph.

The uniqueness of the method comes from combining two distinct approaches ---
Interactive Multiobjective Optimisation (IMO, see~TODO) and Evolutionary
Multiobjective Optimization (EMO, see~TODO).

In the IMO paradigm we want to elicit Decision Maker (DM) preferences by
involving her in the process. This is done by systematic dialog with the
DM. Questions are being asked and the DM provides answers. Based on these
answers preference information is extracted. Algorithm can then use this
knowledge to produce solution better fitted to the Decision Maker
preferences. The IMO framework is presented on
fig.~\ref{fig:interactive-process}.

\begin{figure} 
  \begin{center}
    \begin{tikzpicture}[start chain, node distance=10mm,
        every join/.style={->}]

      
      \node (gen)     [on chain, join, roundrect]{
        \begin{tabular}{c}
          1. Generate set\\
          of possible solutions
      \end{tabular}};

      \node (ask) [on chain, join, roundrect] {
        \begin{tabular}{c}
          2. Ask the DM \\
          to indicate `good' ones
      \end{tabular}};

      \begin{scope}[start branch=store]
        \node (store) [roundrect,on chain=going below left,join] {
          \begin{tabular}{c}
            3. Extract \\
            preference information
        \end{tabular}};
      \end{scope}

      
      \path (store) edge[->] (gen);
    \end{tikzpicture}
    \caption{Typical Interactive Multiobjective
      Optimisation process framework\label{fig:interactive-process}}
  \end{center} 
\end{figure} 



Most of past research on Evolutionary Multiobjective Optimization (EMO) at-
tempts to approximate the complete Pareto-optimal front by a set of well-
distributed representatives of Pareto-optimal solutions. The underlying reason-
ing is that in the absence of any preference information, all Pareto-optimal so-
lutions have to be considered equivalent.


    On the other hand, in most practical applications, the Decision Maker (DM)
is eventually interested in only a small subset of good solutions, or even a single
most preferred solution. In order to come up with such a result, it is necessary
to involve the DM. This is the underlying idea of another multiobjective opti-
mization paradigm: Interactive Multiobjective Optimization (IMO). IMO deals
with the identification of the most preferred solution by means of a systematic
dialogue with the DM. Only recently, the scientific community has discovered
the great potential of combining the two paradigms (see (2)). From the point of
view of EMO, involving the DM in an interactive procedure allows to focus the
search on the area of the Pareto front which is most relevant to the DM. This,
in turn, may allow to find preferred solutions faster. In particular, in the case
of many objectives, EMO has difficulties, because the number of Pareto-optimal
solutions becomes huge, and Pareto-optimality is not sufficiently discriminative
to guide the search into better regions. Integrating user’s preferences promises
to alleviate these problems, allowing to converge faster to the preferred region
of the Pareto-optimal front.
