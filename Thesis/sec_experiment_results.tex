DARWIN is a new method and according to knowledge of the author and the
supervisor haven't been implemented before. Experiments are
designed to check if the method is working at all, what parameters are important
for the method and what should be their reasonable default values. To make
results repeatable the DM is mocked. Noise in his or her decisions is
simulated. Unless uncertainty is involved comparisons to exact optimal
solution are provided. In tests involving uncertainty results are compared to
supposed utility function optimisation. 

\section{The environment}

All tests were conducted on a personal computer with 64bit Intel
processor. RAM size on the machine is 3GB. 64bit Linux operating system was
used. The Java Virtual Machine was in version 1.6.0\_18 and Scala 2.8.0. JVM
was run with options \texttt{-Xms768 -Xmx768} thus setting memory available
for application to 768MB. Tests were performed through CLI batch interface.

Test framework is available in order to automate the experiment process. All
experiments were repeated at least thirteen times. Data analysis and chart
generation was performed using an R environment [ref]. The framework is a
combination of Python [ref] and Bash [ref] code communicating with main DARWIN
code and with the modules written in R.

\section{Problem selection}

Area of interest for Multi-Objective Optimisation is huge and consists of many
potential problems to be solved. There are multi-criteria versions of
classical problems, like minimal spanning tree~[ref], traveling salesman
problem (TSP)~[ref] or knapsack problem~[ref] as well as artificially
generated ones --- like the ZDT problems [ref]. Some of them are interesting
because of their real-life applications while the other are good for
experimenting and testing purposes.

The experiments were performed using following problems:
\begin{enumerate}
  \item Two-criteria binary knapsack problem
  \item Two-criteria continuous knapsack problem
  \item Three-criteria binary knapsack problem
  \item Three-criteria ZDT-problem
\end{enumerate}

\section{No uncertainty}

\section{The importance of parameters}

\section{Noise in the DM's decisions}

\section{Uncertainty}

\section{Conclusions}

